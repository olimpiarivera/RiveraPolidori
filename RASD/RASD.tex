\DeclareUnicodeCharacter{2028}{} 
\documentclass{article}
\usepackage{blindtext}
\usepackage[T1]{fontenc}
\usepackage[utf8]{inputenc}
\usepackage{titling}
\usepackage{tikz}
\usetikzlibrary{calc}
\usetikzlibrary{arrows.meta}
\usepackage{graphicx}


\title{ Requirements Analysis and Specifications Document\\2020-2021}
\author{Alessandro Polidori (Codice persona 10573078)\\Olimpia Rivera (Codice persona 10617517)}
\date{}



\renewcommand{\contentsname}{Table of Contents}


\begin{document}

\maketitle
\tableofcontents{}

\newpage

\section{Introduction}

\subsection{Purpose}
This document provides an analysis of the system in terms of assumptions, functional and non functional requirements needed to fulfill its main goals. It describes the domain in which the system will be deployed by presenting relevant scenarios and use cases and it highlights the software’s limits and constraints.
\bigskip\\
The document is addressed to all the stakeholders affected by the software and is meant to be used by developers in order to realize a system that meets the purpose for which it was intended.\\

\subsection{Scope}
\subsubsection{Description of the given problem}
The CLup system is designed to regulate accesses to stores and manage lines in real time in order to respect restrictions imposed by the virus emergency and avoid crowds and long lines. In particular, the software is offered to both stores managers to monitor the influx of people in their buildings and to common users, allowing them to virtually “line-up” from home and book visits to the stores.\\
\smallskip\\
The main functionalities offered by CLup are the following:\\
\begin{itemize}
\item Basic service: allows stores' customers to line up form home and to approach the store only when their turn is about to arrive. In order for this lining up mechanism to work effectively, the software generates an estimation of the waiting time and alerts users when is the moment to reach the store, taking into account the time they need to get to the shop from the place they were at the time of the reservation. The estimated time is calculated considering the number people both virtually in line and inside the store. The system is also able to recalculate the waiting times of users in the queue and change the status of the line in the event that someone does not show up when her/his turn is arrived or decides to cancel her/his reservation (this mechanism will be further explained in the next sections of the document). In addiction, when customers enter the stores, a QR code generated by the application is scanned, allowing store managers to monitor entrances.
\item Advanced: allows users to book a slot to visit the store. When booking a slot, the customer has to provide the expected duration of the visit or let the software itself infer it (this works only for long-term customers by analyzing the customer’s previous visits). Moreover, the application is able to manage visits in a finer way (taking into account the spaces in the store that customers are going to occupy), by allowing users to indicate the list (or at least the categories) of the items that they intend to purchase.
\end{itemize}
\subsubsection{World Phenomena}
\subsubsection{Shared Phenomena}
\subsubsection{Goals}

\subsection{Definitions, Acronyms, Abbreviations}
\subsubsection{Definitions}
\subsubsection{Acronyms}
\subsubsection{Abbreviations}

\subsection{Reference Documents}
\subsection{Overview}
The RASD document is structured in the following 5 chapters:
\begin{itemize}
\item\textbf{Chapter 1} describes the document’s purpose and brief identification of the context in which the application is going to work given its main functionalities. It also provides lists of world phenomena, shared phenomena and goals that the system is supposed to achieve. Finally, useful specifications (definitions, acronyms, abbreviations) are included for a better understanding of the next sections of the document.
\item\textbf{Chapter 2} gives an overall description of the project. The class diagram in the Product Perspective provides a conceptual overview of the main elements of the system and the state charts describe the evolution of relevant objects. In the Product Function section, the system’s high level functionalities are further detailed and clarified. The expected type of actors that will interact with the system are listed in User Characteristics. Ultimately, chapter 2 describes the system’s constraints and the domain properties assumed to hold in the world.
\item\textbf{Chapter 3}
\item\textbf{Chapter 4}
\item\textbf{Chapter 5}
\end{itemize}


\end{document}